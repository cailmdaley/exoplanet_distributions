
\documentclass[12 pt, letterpaper]{article}
% \usepackage[margin=1in]{geometry}

\begin{document}
\title{On the Relation Between Intrinsic and RV-Traced Exoplanet Distributions}
\author{Cail Daley}
\maketitle
\abstract

The true exoplanet distribution across a given parameter (planet mass, semi-major axis, etc.)  is not, of course, perfectly traced by the distribution derived from radial velocity observations. 
This is due to many factors and biases, in particular the bias towards high-mass planets close to their host star. 
Assuming we understand these biases well, it should be possible to find the “true” exoplanet distribution that corresponds to the distribution traced by RV techniques. 
While such a goal falls outsides the scope of this project, I would like to take inspiration from this concept and explore the relations between intrinsic distributions and those traced by RV techniques.
This would involve creating synthetic radial velocity curves for hundreds/thousands of systems, varying the number of planets, planet masses and separations, etc. Then for each system, I will use some combination of nested sampling and MCMC methods to fit the data, allowing the distribution derived from RV data to be compared to the `true' data. 
I would like to emphasize that the goal of this project does not take the form of an attempt to directly learn about the intrinsic exoplanet distribution in our universe, but rather to learn about the relation between RV-observed and intrinsic distributions, \textit{regardless of the form the distribution takes}.



\end{document}
